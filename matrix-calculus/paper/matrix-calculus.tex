%Matrix Calculus
%HU, Pili
%Craete: 20120330

%Modify the relative path accordingly
%HU, Pili
%Create: 20120330
%Modify: 20120330
%The unified entry to include in my tutorial series

%HU, Pili
%Create: 20110910
%Modify: 20120330
%purpose of this file is to gather commonly used
%mathematical abbreviations, to speed up writing
%notes

\documentclass[11pt,a4paper]{article}
\usepackage[utf8x]{inputenc}
\usepackage{ucs}
\usepackage{amsmath}
\usepackage{amsfonts}
\usepackage{amssymb}
\usepackage{amsthm}
\usepackage{url}
\usepackage{graphicx}

\usepackage{fancyhdr}
\pagestyle{fancy}
\fancyhead{}

%=====Calculus======
%the following commands are not originated by me
%I pick them from http://www-solar.mcs.st-and.ac.uk/~clare/Latex/
%the following line controls the style of patial derivative
%1), use \dfrac, height is larger, looks good. 
%2), use \frac, also work, but space looks limited. 
\newcommand{\myfrac}[2]{\dfrac{#1}{#2}}
\newcommand{\diff}[2]{\myfrac{{\rm d}#1}{{\rm d}#2}}
\newcommand{\ndiff}[3]{\myfrac{{\rm d}^{#3}#1}{{\rm d}#2^{#3}}}
\newcommand{\pdiff}[2]{\myfrac{\partial #1}{\partial #2}}
\newcommand{\npdiff}[3]{\myfrac{\partial^{#3} #1}{\partial #2^{#3}}}
\newcommand{\e}[1]{\ensuremath{{\rm e}^{#1}}}
\newcommand{\ldiff}[2]{\ensuremath{{\rm d}#1/{\rm d}#2}}
\newcommand{\lpdiff}[2]{\ensuremath{\partial#1/\partial#2}}
\newcommand{\lnpdiff}[3]{\ensuremath{\partial^{#3}#1/\partial#2^{#3}}}
\newcommand{\dif}[1]{\mathrm{d}#1}

%20120330
%The reason I don't copy the original file as a whole
%is that it contains too many individually preferred 
%definitions. 
%
%I start with those basic symbols and adapt them in use. 

%=====Matrix======
\newcommand{\tr}[1]{\mathrm{Tr}\left[#1\right]}
\newcommand{\tran}[1]{#1^\mathrm{T}}
%The following shorthand of matrix may be convenient. 
%However, it is so short that I'm worried it may 
%collide with something else. I don't use at present.
%\newcommand{\m}[1]{\mathbf{#1}}
\newcommand{\adj}[0]{\mathrm{adj}}

%=====Theorem definitions=====
\newcounter{mytheoremorder}
\newtheorem{mydef}{Definition}
\newtheorem{myaxm}{Axiom}
\newtheorem{mythm}[mytheoremorder]{Theorem}
\newtheorem{myprop}[mytheoremorder]{Proposition}
\newtheorem{myex}{Example}

%=====Optimization====
\DeclareMathOperator*{\argmax}{arg\,max}
\DeclareMathOperator*{\argmin}{arg\,min}
\newcommand{\maximize}[0]{\mathrm{Maximize~}}
\newcommand{\minimize}[0]{\mathrm{Minimize~}}

%=====Probability====
\newcommand{\E}[0]{\mathbb{E}}
\newcommand{\var}[0]{\mathrm{Var}}
\newcommand{\cov}[0]{\mathrm{Cov}}



%This usually doesn't need modification 
\author{HU, Pili\thanks{hupili [at] ie [dot] cuhk [dot] edu [dot] hk}}

%Modify them accordingly===
\title{Matrix Calculus: \\ Derivation and Simple Application}
\date{March 30, 2012\thanks{Last compile:\today}}

\begin{document}

\maketitle
%>============================================
\begin{abstract}
	Matrix Calculus\cite{wiki_mc} is a very useful tool in many 
	engineering problems. Basic rules of matrix calculus are 
	nothing more than ordinary calculus rules covered in 
	undergraduate courses. However, using matrix calculus, 
	the derivation process is more compact. This document is 
	adapted from the notes of a course the author recently attends.
	It builds matrix calculus from scratch. Only prerequisites 
	are basic calculus notions and linear algebra operation.  
	To get a quick executive guide, please refer to the cheat 
	sheet in the end. 
\end{abstract}
%<=======Abstract ENd=========================

%>============================================
\pagebreak
\tableofcontents
\pagebreak
%<=======TOC ENd==============================



\section{Introductory Example}

We start with an one variable linear function:
\begin{equation}
	f(x) = ax
\end{equation}

To be coherent, we abuse the partial derivative notation:
\begin{equation}
	\pdiff{f}{x} = a
	\label{eq:fax-single}
\end{equation}

Extending this function to be multivariate, we have:
\begin{equation}
	f(x) = \sum_{i}{a_ix_i} = \tran{a}x
\end{equation}
%The followings are the suggested transpose online? 
%What do you prefer? 
%I choose \mathrm at present. It looks reasonably good. 
%	a^\intercal
%	a^\mathsf{T}
%	a^\mathrm{T}
%	a^\top
%	a^\bot
Where $a = \tran{[a_1,a_2,\ldots,a_n]}$ and 
$x = \tran{[x_1,x_2,\ldots,x_n]}$. 
We first compute partial derivatives directly:
\begin{equation}
	\pdiff{f}{x_k} = \pdiff{(\sum_{i}{a_ix_i})}{x_k} = a_k 
\end{equation}
for all $k=1,2, \ldots, n$. Then we organize $n$ partial derivatives
in the following way:
\begin{equation}
	\pdiff{f}{x} = \left[
	\begin{matrix}
		\pdiff{f}{x_1} \\
		\pdiff{f}{x_2} \\
		\vdots \\
		\pdiff{f}{x_n}
	\end{matrix}
	\right]
	= \left[
	\begin{matrix}
		a_1 \\
		a_2 \\
		\vdots \\
		a_n
	\end{matrix}
	\right]
	= a
	\label{eq:fax-multi}
\end{equation}
The first equality is by proper definition and the rest roots from 
ordinary calculus rules. 

Eqn(\ref{eq:fax-multi}) is analogous to eqn(\ref{eq:fax-single}), except
the variable changes from a scalar to a vector. Thus we want to directly 
claim the result of eqn(\ref{eq:fax-multi}) without those intermediate steps 
solving for partial derivatives separately. Actually, we'll see soon 
that eqn(\ref{eq:fax-multi}) plays a core role in matrix calculus. 

Following sections are organized as follows:
\begin{itemize}
	\item Section(\ref{sec:derivation}) builds commonly used 
	matrix calculus rules from ordinary calculus and linear 
	algebra. Necessary and important properties of linear 
	algebra is also proved along the way. 
	\item Section(\ref{sec:application}) shows some applications 
	using matrix calculus. 
	\item Section(\ref{sec:cheat}) concludes a cheat sheet of 
	matrix calculus. Note that this cheat sheet may be different 
	from others. Users need to figure out some basic definitions 
	before applying the rules. 
\end{itemize}



\section{Derivation}
\label{sec:derivation}

\section{Application}
\label{sec:application}

\subsection{The 2nd Induced Norm of Matrix}

\section{Cheat Sheet}
\label{sec:cheat}


%>============================================
\section*{Acknowledgements}
\addcontentsline{toc}{section}{Acknowledgements}
Thanks prof. XU, Lei's tutorial on matrix calculus. 
Besides, the author also benefit a lot from other online 
materials. 
%<=======Acknowledgements ENd=================

%>============================================
\addcontentsline{toc}{section}{References}
\input{../reference/gen_bib.bbl}
%<=======Bibliography ENd=====================

%>============================================
\section*{Appendix}
\addcontentsline{toc}{section}{Appendix}

%<=======Appendix ENd=========================

\end{document}
